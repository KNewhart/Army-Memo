%%% LOAD PACKAGES %%%

% Warn when deprecated LaTeX commands are used
\RequirePackage[l2tabu, orthodox]{nag}

\documentclass[12pt,letterpaper]{article}

% Make nice clickable urls
\usepackage[pdftex, pdfusetitle,colorlinks=false,pdfborder={0 0 0}]{hyperref}%

% Include eps files
\usepackage{graphicx}

% Set margins easily
\usepackage[margin=1in]{geometry}

\usepackage{setspace}

% Use expanded computer modern that has bold small caps
\usepackage[T1]{fontenc}

\usepackage{tikz}

\usepackage{microtype}

% more control over enumerate
\usepackage{enumitem}

% arbitrary fontscaling with \scalefont
\usepackage{scalefnt}

% Generate dummy text
\usepackage[english]{babel}
\usepackage{blindtext}

%%% FORMATTING %%%

% Helvetica for the letterhead.  AR 25-50 says use arial, AR 25-40
% says helvetica.
%\usepackage[scaled]{helvet}

% Times for the body.  Everyone expects Times for the body copy.
% However, AR 25-50 says to use Arial.
%\usepackage{times}
\usepackage{fontspec}
\setmainfont{Arial}

% Select date
%\usepackage[dmmyyyy]{datetime}
%\renewcommand{\dateseparator}{ }
\usepackage{datetime2}
\makeatletter
\newcommand{\daymonthyear}{%
  \@dtm@day\space\DTMenglishshortmonthname{\@dtm@month} \@dtm@year
}
\makeatother

% Department of the Army command
\newcommand{\daletterhead}[1]{ %
  \fontsize{10pt}{10pt}\selectfont %
  \textbf{\textsc{#1}}\\%
}

\newcommand{\letterhead}[1]{ %
  \fontsize{8pt}{8pt}\selectfont %
  \textbf{\textsc{#1}}\\%
}

% Superscripts, 1st, 2nd, etc...
\newcommand{\st}{\textsuperscript{st}}
\newcommand{\nd}{\textsuperscript{nd}}
\newcommand{\rd}{\textsuperscript{rd}}
\newcommand{\thh}{\textsuperscript{th}}

\newcommand{\memotype}[1]{\noindent\textsc{#1}}
\newcommand{\subject}[1]{\noindent\textsc{Subject}: #1}

\newcommand\textbox[1]{%
  \parbox{.5\textwidth}{#1}%
}


% Numbering
%
% Manually:
%    \begin{enumerate}[labelwidth=12pt,labelsep=4pt,leftmargin=0pt,itemindent=0.5in,align=left]
%        \begin{enumerate}[leftmargin=0pt,itemindent=1in,align=left]
              % AR 25-50, Figure 2-1 specifices that the fourth item
              % is indented the same amount as the third item.  No
              % additional list levels are allowed.
%              \begin{enumerate}[leftmargin=0pt,itemindent=1in,align=left]
%              \end{enumerate}
%        \end{enumerate}
%    \end{enumerate}
%\end{enumerate}
%
% Command (must remember to include \end{enumerate}:
\newcommand{\beginEnumerateOne}{\begin{enumerate}[labelwidth=12pt,labelsep=4pt,leftmargin=0pt,itemindent=16pt,align=left]}
\newcommand{\beginEnumerateTwo}{\begin{enumerate}[labelwidth=12pt,labelsep=4pt,leftmargin=0pt,itemindent=0.5in,align=left]}
\newcommand{\beginEnumerateThree}{\begin{enumerate}[leftmargin=0pt,itemindent=1in,align=left]}
\newcommand{\beginEnumerateFour}{\begin{enumerate}[leftmargin=0pt,itemindent=1in,align=left]}

% Acronyms - AR 25-52: Authorized Abbreviations, Brevity Codes, and Acronyms
% 
% Other locations for acronyms: JP1-02: DOD Dictionary of Military and
% Associated Terms, FM 1-02: Operational Terms and Graphics
\usepackage{glossaries}
\usepackage{relsize}
\setacronymstyle{long-sm-short}
% Example:
%\newacronym{dod}{DOD}{Department of Defense}

% Setup list style
\renewcommand{\theenumi}{\arabic{enumi}}
\renewcommand{\labelenumi}{\theenumi.}

\renewcommand{\theenumii}{\alph{enumii}}
\renewcommand{\labelenumii}{\theenumii.}

\renewcommand{\theenumiii}{\arabic{enumiii}}
\renewcommand{\labelenumiii}{(\theenumiii)}

\renewcommand{\theenumiv}{\alph{enumiv}}
\renewcommand{\labelenumiv}{(\theenumiv)}

\usepackage{fancyhdr}
% No line in header
\renewcommand{\headrulewidth}{0pt}
\pagestyle{fancy}
% Nothing on the firstpage.  To use, place the following somewhere on
% the first page
%
% \thispagestyle{firststyle}
\fancypagestyle{firststyle} {\fancyhf{} \fancyfoot{}}

\lhead{\scriptsize\OfficeSymbol\\ \footnotesize \textsc{subject}: \Subject}
\fancyfoot[C]{\thepage}

%%% INPUTS %%%
\newcommand{\Author}{Kathryn B. Newhart, Ph.D.}
%\newcommand{\Subject}{EV401 Course Letter 22-2}
\newcommand{\OfficeSymbol}{MADN-GEN}
\newcommand{\Date}{\daymonthyear}
\newcommand{\Keywords}{.}


%%% SETUP %%%
\hypersetup{%
pdftitle={\Subject},%
pdfauthor={\Author},%
pdfkeywords={\Keywords},%
}%

\usepackage{booktabs}
\usepackage{float}

\begin{document}

% Position DOD seal absolutely, AR 25-40, Appendix G-3. Letterhead and
% memorandum stationery.  1/2in from the top edge of the page and 1/2in
% from the left edge of the page
%
% Regarding other logos, AR 25-30, 7-7.a states, "Do not print any
% seals, emblems, decorative devices, distinguishing insignia,
% slogans, office symbols, or mottos on letterhead or memorandum
% stationery except those approved or directed by HQDA"
\begin{tikzpicture}[remember picture, overlay]
  \node[anchor=north west,inner sep=0pt,xshift=0.5in, yshift=-0.5in] at (current page.north west) { %
    \includegraphics[width=1in]{dod_seal_black.pdf}%
  };%
\end{tikzpicture}%

% Positioning the letterhead.
%
% AR 25-40, Appendix G-3, figure G-1 (text on the side of the figure):
%
% "Heading is centered on the page 5/8 inch from the top trim. All
% type is Helvetica bold. 'Department of the Army' is 10pt, all other
% type is 8 point except 'reply to', 'attention of', which is 6
% point. (Use of the phrase 'reply to, attention of' is optional.)
% The DOD seal is 1 inch in diameter"
%
% AR 25-30, 7-7.c covers the format of the letterhead: "Include the
% complete street address and ZIP+4 Code."  But then it says refer to
% AR 25-40, figure G-1 and G-2.
\begin{tikzpicture}[remember picture, overlay]
  % Use helvetica
  \setstretch{.75}
  %\fontfamily\sfdefault\selectfont
  \node[align=center,anchor=north,inner sep=0pt,yshift=-0.625in] at (current page.north) {
    \daletterhead{Department of the Army}
    \letterhead{United States Military Academy}
    \letterhead{Department of Geography \& Environmental Engineering}
    \letterhead{West Point, New York 10996-1695}
  };%
\end{tikzpicture}%

% TODO: figure out where the page starts, we should be 12pts below the
% bottom of the DOD seal.
\vspace{15pt}

% Scale the office symbol down a bit so it's not the focus point
\noindent\textbox{\scalefont{.9}\OfficeSymbol\hfill} \textbox{\hfill\Date}

\vspace{24pt}

\memotype{\Memo}

\vspace{12pt}

\subject{\Subject}

\vspace{12pt}

\thispagestyle{firststyle}

% AR 25-50 2-3.(2).c - Margins: "Do not right justify margins."
\raggedright

\setlength{\parindent}{0.5in}
\setlength{\parskip}{1em}

%%% BODY %%%
% Directly input or import from file:
%\input{EV401-courseletter-222}
Testing.
